\chapter{Introduction}\label{ch:introduction}
\section{Background}\label{sec:introduction:background}
Whenever I talk to someone about the topic of my thesis, I receive a confused look followed by a question:
\begin{quotation}
   I understand the \textit{mobile application}, but what the \textit{gamification} is?
\end{quotation}

As an answer, I proudly quote the most popular and widely accepted definition, which I know by heart already:
\begin{quotation}
    It is the use of game design elements in non-game contexts \cite{deterdingGamificationDefinition2011}.
\end{quotation}
The look of confusion on my counterpart's face usually grows even bigger, so I quickly follow up with a simpler explanation:
\begin{quotation}
    You certainly like playing games. There are mechanisms that cause it to be fun and make you want to play it. You can play with your friends, collect points and claim rewards - those are \textit{game design mechanisms}. And \textit{non-game contexts}? It is essentially everything but a game and using these mechanisms in every place outside of a game - may it be sports, productivity, or even health.
\end{quotation}
A focused frown is immediately replaced with a self-satisfied smile. I hope it also came to The Reader's face.
\\\\
One of the earliest applications of gamification reaches back to the early 1910s when the Scouts movement was found. Since then, their flagship programme was to reward young adepts with badges for various activities. Moreover, it is a flagship example of gamification itself, long before the definition was formed \cite{geraldchristiansOriginsFutureGamification,deterdingGamificationDesigningMotivation2012}. A few decades later, in 1973, Charles A. Coonradt noticed that in sports, unlike in the United States' workforce, productivity and teamwork was blooming. His attempts to transfer the same strategies to the business environment resulted in a methodology called \textit{The Game of Work} (which he later described in a book with the same title \cite{coonradtGameWork2012}). This was groundbreaking, as proved that gamification is not only a way to entertain children, but can also be profitable in the commercial world. 
\\\\
At present, there are countless examples of effective gamification implementations, not only in business \cite{hamariDoesGamificationWork2014} or crowdsourcing \cite{morschheuserGamificationCrowdsourcingReview2016}, but also in healthcare \cite{pereiraReviewGamificationHealthRelated2014} and, most importantly in the context of this thesis, child development \cite{kondracka-szalaGamesDevelopingSelected2017, nandEngagingChildrenEducational2019}.


\section{Motivation}\label{sec:introduction:motivation}
\say{Gaming disorder}. The name was given to an addictive behaviour related to gaming by the World Health Organisation in 2018\cite{worldhealthorganisationAddictiveBehavioursGaming}. It is \say{characterized by impaired control over gaming, increasing priority given to gaming over other activities to the extent that gaming takes precedence over other interests and daily activities [...]}. It results in \say{significant impairment in personal, family, social, educational, occupational or other important areas of functioning [...]}. In 2019 Society for Research in Child Development reported that in United Kingdom 66\% of children aged 5-7 regularly play games for around 7.5 hours a week and around 80\% of those aged 8-15 play for at least 10 hours a week \cite{hygenTimeSpentGaming2020}.
\\\\
Newzoo, one of the most reliable sources of the game market analytics, estimates that in 2020 the game industry revenue will reach \$160 billion, almost half of which will be gaming on mobile devices. In consequence of the 10\% year-to-year growth, by 2023  annual revenue will have exceeded \$200 billion worldwide \cite{tomwijmanWorldBillionGamers2020}. And COVID-19 pandemic is only increasing these numbers.
\\\\
Such statistics make one appreciate the scale of the problem and its impact on the young generation. It also provokes thoughts about how the situation could be improved. If only there was a way to copy the mechanisms that cause the games to be so appealing and use them to motivate children to be more productive...
\\\\
I hope the smile comes back to The Reader's face now as they remember the solution described in the previous paragraph - \textit{gamification}. And probably the best way to implement it is by using the most popular medium among youth - mobile devices. Therefore the topic of this thesis - \say{Mobile application with gamification elements}. 


\section{Goal}\label{sec:introduction:goal}
The goal of this thesis is to develop a project of a mobile application with gamification mechanisms, to implement a prototype of a working product and verify it.


\section{Scope of Work}\label{sec:introduction:scope}
// TODO: Implement in the end


\section{Acknowledgements}\label{sec:introduction:acknowledgments}
// TODO: Implement in the end