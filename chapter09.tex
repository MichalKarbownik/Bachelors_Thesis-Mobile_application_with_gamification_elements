\chapter{Work summary}\label{ch:summary}

\section{Conducted work}\label{sec:summary:results}

% static code analysis was not performed before every commit (the way it should), instead it was only checked in CI/CD in order to measure how many builds fail

% underestimated tickets

% \section{Achieving the Project Objectives}
The objective of the work was to develop a project of a mobile application with gamification mechanisms, to implement a prototype of a working product and verify it.
\\\\
The work was preceded by shaping project assumptions. It helped with visualising a bigger picture of the product that was going to be created. Market analysis ensured that the existing solutions left a niche that needed to be filled. Forming project requirements established guidelines for further work. Created project and tools selection provided a foundation for a well-executed implementation phase. Finally, tests validated the product's proper operation and successful delivery.
\\\\
Having said that, the project goal can be considered as achieved. Even though, there is room for improvements, the application has been completed and is already used by the targeted audience. The gamification elements in the form of tasks, points and rewards help parents to boost their children's productivity.

\section{Further work suggestions}\label{sec:summary:furhter}
Despite the system being highly rated by the first group of users, it still has a room for improvement. In this project, there are three established sources of gaining knowledge about future improvements:
\begin{itemize}
    \item implementation challenges (described in Section \ref{sec:implementation:challenges}),
    \item user needs (revealed in Section \ref{sec:tests:user}),
    \item business needs (discussed in Chapters \ref{ch:assumptions}, \ref{ch:market} and \ref{ch:requirements}),.
\end{itemize}
All of them will be briefly analysed in the following sections.


\subsection{Implementation challenges}
Even though during the development, standard conventions and guidelines have been followed, the application's code needs to go through a refactor process. While the naming and structure on the macro level are acceptable, it is the micro level that needs additional attention.
\\\\
Moreover, an alternative solution for both, the \texttt{Canvas} class used for the header composition, and the \texttt{Provider} package for the state management, should be considered. While their choice was backed by deep research and analysis, they did not fulfill the needs sufficiently.


\subsection{User needs}
Opportunities to improve in relation to user needs were discovered during the user acceptance tests. The remarks that happened the most often concerned the understandability and learnability of the application. The most important aspects regarded the onboarding process and multiple language support - these should be taken into account in the future iterations.


\subsection{Business needs}
Business needs often overlap with user needs. Sometimes, however, additional steps are expected to be taken in order to meet them. The initial projects did not any other kinds of measuring users' behaviour and satisfaction than questionnaires. Heatmaps, A/B testing, in-app reviews and gathering more data about the application itself are suggested.

\section{Conclusions}\label{sec:summary:conclusions}
It is important to remember that the generation of children, whose future we do shape now, will once shape our future. Utilizing technology in this process is inevitable. However, if only we use adequate tools, we can change the future, our common future, for the benefit of all.
\\\\
As a closing note, I would like to quote the pioneer of the modern education of children, who also happens to be a patron of the kindergarten I used to attend, Maria Montessori:
\begin{quote}
Children are human beings to whom respect is due, superior to us by reason of their innocence and of the greater possibilities of their future… Let us treat them with all the kindness which we would wish to help to develop in them.
\end{quote}
