\newcommand{\neworrenewcommand}[1]{\providecommand{#1}{}\renewcommand{#1}}

\newcommand{\UseCaseTable}[9]{%
\neworrenewcommand{\UseCaseTableHelper}[2]{%
% #1 - UC number
% #2 - UC identifier
% #3 - UC name
% #4 - Primary Actor
% #5 - Trigger
% #6 - Preconditions
% #7 - Postconditions
% #8 - Basic flow
% #9 - Alternative flows
% ##1 - Exceptions
% ##2 - Extensions
\begin{table}[htb]
\centering
\begin{tabular*}{\linewidth}
{p{0.2\linewidth}p{0.74\linewidth}}
\toprule
Use Case & UC#1
\\
\midrule 
Name & #3
\\
\midrule 
Primary actor & #4
\\
\midrule 
Trigger & #5
\\
\midrule 
Preconditions & #6
\\
\midrule 
Postconditions & #7
\\
\midrule 
Basic flow & #8
\\
\midrule 
Alternative flows & #9
\\
\midrule 
Exceptions & ##1
\\
\midrule 
Extensions & ##2
\\
\bottomrule
\end{tabular*}
\caption{\textit{#3} use case description}
\label{tab:use-case:#2}
\end{table}
}
\UseCaseTableHelper
}