\begin{table}[htb]
\centering
\begin{tabular*}{\linewidth}
{p{0.23\linewidth}p{0.35\linewidth}p{0.35\linewidth}}
\toprule
\multicolumn{1}{c}{Aspect} & \multicolumn{2}{c}{Technology} 

\\

\cmidrule{2-3} 
& \multicolumn{1}{c}{Flutter}
& \multicolumn{1}{c}{React Native} 

\\

\midrule

UI components 

& Default package provides numerous components with consistent look across platforms (able to imitate platform-specific look).

& Uses native components, gives more flexibility in terms of customisation, but takes longer to achieve some results.

\\

Package library

& Separation of verified \textit{publishers} and unverified \textit{uploaders} on \textit{pub.dev}\tablefootnote{Flutter package registry publishing guide, https://pub.dev/help/publishing (accessed Dec. 06, 2020).}$^{,}$\tablefootnote{Flutter package registry verification process,  https://dart.dev/tools/pub/verified-publishers (accessed Dec. 06, 2020).}, resulting in a better packages quality.

& Anyone being able to publish to the \textit{npm registry}\tablefootnote{Node Package Manager registry publishing guide, https://docs.npmjs.com/about-the-public-npm-registry (accessed Dec. 06, 2020).}, which sometimes results in lower quality and information overload.

\\

Documentation

& Extensive, detailed, centralized. 

& Not as extensive, detailed, decentralized (responsibilities shifted onto particular packages). 

\\

Development

& Custom, fresh ecosystem, but providing just enough tools.

& Utilizing extensive web ecosystem improved for years.

\\

Maturity

& Quite young, yet mature; growing rapidly. 

& Mature and growing.

\\
\bottomrule

\end{tabular*}
\caption{Comparison of \textit{React Native} and \textit{Flutter}}
\label{tab:tools:architecture:application:flutter-react}
\end{table}